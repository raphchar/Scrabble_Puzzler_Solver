

% 
% Instructions for using this template:
% This title page is capable of being compiled as is. This is not useful for 
% including it in another document. To do this, you have two options: 
%
% 1) Copy/paste everything between \begin{document} and \end{document} 
% starting at \begin{titlepage} and paste this into another LaTeX file where you 
% want your title page.
% OR
% 2) Remove everything outside the \begin{titlepage} and \end{titlepage}, rename
% this file and move it to the same directory as the LaTeX file you wish to add it to. 
% Then add \input{./<new filename>.tex} to your LaTeX file where you want your
% title page.
%
%%%%%%%%%%%%%%%%%%%%%%%%%%%%%%%%%%%%%%%%%

%----------------------------------------------------------------------------------------
%	PACKAGES AND OTHER DOCUMENT CONFIGURATIONS
%----------------------------------------------------------------------------------------

\documentclass[11pt]{article}

\usepackage[utf8]{inputenc} % Required for inputting international characters
\usepackage[T1]{fontenc} % Output font encoding for international characters
\usepackage{fancybox}
\usepackage{tikz}
\usepackage{mathpazo} % Palatino font
\usepackage{graphicx}
%\usepackage[english]{babel}
\usepackage[francais,english]{babel}
\usepackage{amsmath}
\usepackage{algorithm}
\usepackage[noend]{algpseudocode}
\usepackage{xparse}
\usepackage{mathtools}
\DeclarePairedDelimiter\ceil{\lceil}{\rceil}
\DeclarePairedDelimiter\floor{\lfloor}{\rfloor}

\usepackage{listings}
\lstset{ 
	language=Matlab,                		% choose the language of the code
%	basicstyle=10pt,       				% the size of the fonts that are used for the code
	numbers=left,                  			% where to put the line-numbers
	numberstyle=\footnotesize,      		% the size of the fonts that are used for the line-numbers
	stepnumber=1,                   			% the step between two line-numbers. If it's 1 each line will be numbered
	numbersep=5pt,                  		% how far the line-numbers are from the code
%	backgroundcolor=\color{white},  	% choose the background color. You must add \usepackage{color}
	showspaces=false,               		% show spaces adding particular underscores
	showstringspaces=false,         		% underline spaces within strings
	showtabs=false,                 			% show tabs within strings adding particular underscores
%	frame=single,	                			% adds a frame around the code
%	tabsize=2,                				% sets default tabsize to 2 spaces
%	captionpos=b,                   			% sets the caption-position to bottom
	breaklines=true,                			% sets automatic line breaking
	breakatwhitespace=false,        		% sets if automatic breaks should only happen at whitespace
	escapeinside={\%*}{*)}          		% if you want to add a comment within your code
}

\begin{document}

%----------------------------------------------------------------------------------------
%	TITLE PAGE
%----------------------------------------------------------------------------------------

\begin{titlepage} % Suppresses displaying the page number on the title page and the subsequent page counts as page 1
	\newcommand{\HRule}{\rule{\linewidth}{0.5mm}} % Defines a new command for horizontal lines, change thickness here
	
	\center % Centre everything on the page
	
	%------------------------------------------------
	%	Headings
	%------------------------------------------------

	
	\textsc{\Large  INFO0027 : Programming Techniques}\\[0.5cm] % Major heading such as course name
	
	
	%------------------------------------------------
	%	Title
	%------------------------------------------------
	
	\HRule\\[0.4cm]
	
	{\huge\bfseries Project 1 - Scrabble Puzzle Solver}\\[0.4cm] % Title of your document
	
	\HRule\\[1.5cm]
	
	%------------------------------------------------
	%	Author(s)
	%------------------------------------------------
	
	
		\begin{center}
		
			\large
			\textsc{Lowette} Quentin  \\
			\textsc{Charron} Raphaël \\
		\end{center}
	
	\begin{figure}
  		\includegraphics[width=\linewidth]{ULG_Logo.jpg}
 		\label{fig:boat1}
	\end{figure}

	% If you don't want a supervisor, uncomment the two lines below and comment the code above
	%{\large\textit{Author}}\\
	%John \textsc{Smith} % Your name
	
	%------------------------------------------------
	%	Date
	%------------------------------------------------
	
	\vfill\vfill\vfill % Position the date 3/4 down the remaining page
	
	{\large\today} % Date, change the \today to a set date if you want to be precise
	
	%------------------------------------------------
	%	Logo
	%------------------------------------------------
	
	%\vfill\vfill
	%\includegraphics[width=0.2\textwidth]{placeholder.jpg}\\[1cm] % Include a department/university logo - this will require the graphicx package
	 
	%----------------------------------------------------------------------------------------
	
	\vfill % Push the date up 1/4 of the remaining page
	
\end{titlepage}

%----------------------------------------------------------------------------------------

\tableofcontents
\newpage



\end{document}
